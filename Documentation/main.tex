\documentclass[12pt]{report}
\usepackage[margin=1.0in]{geometry}
\usepackage[utf8]{inputenc}
\usepackage{mathptmx}

\title{Personal Statement}
\author{herna383}
\date{October 2019}

\begin{document}
% Question
% \textbf{Tell us about your current and near-term career-related activities and goals, as well as why you decided to pursue the specific graduate program(s) and school(s) that you have. How do you see your current work and study informing your early career goals? If you have not been accepted into a program yet, please tell us about why you selected the programs to which you are applying.}

My curiosity for understanding why and how things work has been a part of me from a very early age. 
I remember my preschool and kindergarten teachers telling my parents that I asked the most questions out of every kid in my class. 
They expressed to my parents that this habit, if nurtured correctly, could benefit me educationally.
Throughout my academic career, this curiosity drove me to understand class material at a high level.
Eventually, mastering class material was not enough, so I immersed myself into research.
My accumulation of experiences in research, outreach, and helping incoming college students with their college transitions inspired me to become a professor who runs a research lab analyzing and creating new multi-objective optimization techniques.

% Summer research experience with Dr. Weng and challenges overcame
My \textbf{first academic research experience} was working with Dr. Juyang Weng on a summer project involving artificial neural networks (ANNs).
I was tasked with developing a system that could visualize, monitor, and analyze memory usage and classification accuracy for an ANN developed by Dr. Weng. 
Dr. Weng was traveling abroad with minimal means of communication, which impacted the amount of guidance I received.
I took initiative to learn as much as I could, as this field was new to me, and sought out books, papers, graduate students, and online tutorials.
Synthesizing this gathered information gained me knowledge on how and why ANNs work, which proved to be useful for tracking the amount of memory it consumes and classification accuracy.
By the end of the summer I fully developed a interface and integrated the ANN that needed to be analyzed, which helped users tune parameters required by the ANN.
This interface provided users a tool to help tune hyperparameters required from an ANN.  
The thought of using ANNs to solve complicated problems left me excited to explore other Artificial Intelligence techniques even further. 

% Summer research experience with Charles and how I felt more confident 
During an algorithm engineering lecture, Dr. Charles Ofria was presenting his work in evolutionary computation, showing us how digital organisms evolved to solve difficult tasks. 
This application of evolution fascinated me so much that I pursued it in graduate school as a Ph.D. student advised by Dr. Ofria.
My \textbf{second academic research experience} consisted of a summer project using genetic programming to evolve distributed algorithms with Dr. Ofria.
This project challenged me to immerse myself within two new fields: evolutionary computation and distributed systems.
Again, I sought after literature and tutorials that would help familiarize me with the fields.
I felt more confident conducting research and took more initiative during the course of the project. 
I set meetings with professors to discuss ideas regarding my project and worked more independently.
My research skill set expanded by being able to deduce when it is appropriate to ask for advice from an advisor and effectively synthesizing information to utilize in my research.
Being able to see computer programs evolve into effective distributed algorithms in real time left me fascinated to continue conducing research in evolutionary computation.
This project yielded satisfactory distributed algorithms, but could potentially discover more robust ones. 
Both projects were presented at the Mid-Michigan Symposium for Undergraduate Research Experiences and Summer Research Opportunities Program presentations. 

% Summary of what happened and how I have grown as a researcher
After my summer research experiences, I produced two publications with Dr. Ofria. 
I began taking initiative on constructing research questions that interested me. 
I am currently working on a project analyzing multi-objective optimization techniques. 
I also mentor two undergraduate students working with me on this project. 
The project emerged from several wide-ranging conversations between myself, Dr. Ofria, and several other graduate students.
I am proud of leading and independently refining our experimental design. 
I look forward to embracing my intellectual independence and helping my students develop this as a professor. 

% What do I wanna research
My research experiences in Artificial Intelligence sparked an interest in solving complex problems, and my time in Evolutionary Computation demonstrated on of many approaches practitioners use to solve these problems. 
Multi-objective optimization attempts to optimize problems that consist of more than one competing objectives that there exists no globally optimal solution.
Instead it aims to find the solutions that best balance these trade-offs, otherwise known as the Pareto Front.
A real example of a multi-objective optimization problem is trying to decide the best car to buy.
Although it sounds like a simple task, there are many things to consider when buying a car: cost, comfort, performance, fuel efficiency, safety, etc. 
With so many objectives to consider, it can seem impossible find the best car, yet we still manage to.

% What do I plan on doing? 
Currently, multiple multi-objective optimization algorithms exists, all of which consist of different approaches when solving a problem.
Some of these algorithms include $\epsilon$-constraint, weighted sum, lexicographic, NSGA II, etc.
They all perform differently on problems, and I am interested in investigating why these differences occur. 
I propose to explore the strengths and weaknesses behind different multi-objective optimization algorithms. 
This analysis will allow us to construct profiles that can be used to determine the best multi-objective optimization algorithm to use for a given problem.
There is also potential to use this knowledge to create more robust algorithms and classify certain problems. 

% Talk about why msu is the best place for you. BEACON Center, EEBB, Professors here, Charles Ofria, Lab Culture.
MSU is the best place to facilitate my research in multi-objective optimization, as there are an abundant amount of resources. 
I can potentially collaborate with Dr. Kalyanmoy Deb, Dr. Wolfgang Banzhaf, and Dr. Charles Ofria, who are all world-renowned researches in the field of evolutionary computation and have experience working with multi-objective optimization algorithms. 
Along with this, I am a member of the NSF BEACON Center for the Study of Evolution in Action. 
BEACON promotes interdisciplinary research and puts me in contact with people relevant to my research.
These resources not only give me the best chance of executing my research goals, but facilitate an environment where I can thrive and grow. 
\end{document}